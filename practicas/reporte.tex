\documentclass[12pt, letterpaper]{article}
\usepackage[utf8]{inputenc}
\usepackage[left = 2.5cm, right = 2.5cm, top = 3cm, bottom = 3cm]{geometry}
\usepackage[T1]{fontenc}
\usepackage{amsthm}
\usepackage{amsfonts}
\usepackage{amsmath}
\usepackage{amssymb}

\title{Práctica 6: Máquina $\mathcal{K}$ \\
       {\small Lenguajes de Programación I}}

\author{López Soto Ramses Antonio \\
        Quintero Villeda Erik}

\begin{document}
    \maketitle

    \section*{Introducción}

     \subsection*{Objetivo}
        Extender el lenguaje $EAB$, agregando pila y memoria para que se simule la máquina $\mathcal{K}$.

    \section*{Desarrollo}
    A partir de la práctica pasada, extendimos el lenguaje $EAB$, para que simulara la máquina $\mathcal{K}$.
    Los módulos modificados son:

    \begin{center}
        data State = E (Memory, Stack, Expr) | R (Memory, Stack, Expr) | P (Memory, Stack, Expr) deriving (Eq,Show)
    \end{center}

    \begin{itemize}
        \item Sintax
            \begin{itemize}
                \item frVars :: State -> [Identifier]
                \item subst :: State -> Substitution -> State
                \item alphaEq :: State -> State -> Bool
            \end{itemize}

        \item Dynamic
            \begin{itemize}
                \item eval1 :: State -> State
                \item evals :: State -> State
            \end{itemize}
    \end{itemize}

    \section*{Conclusión}
    La práctica fue divertida de impplementar, pues como se dijo en prácticas pasadas, "Dynamic" 
    fue lo que más nos costó trabajo. Tuvimos problemas al implementar las reglas de todo lo
    agregado a "Dynamic", pero se logró arreglar en parte. "Sintax" fue muy sencillo pues al no
    afectar en nada la semántica, fue muy rápida su implementación. \vspace{.3cm}

    En resumen, el objetivo se alcanzó de forma parcial, pero logramos entender de una mejor manera
    cómo es el funcionamiento de la máquina $\mathcal{K}$ junto con las excepciones y/o continuaciones.

\end{document}