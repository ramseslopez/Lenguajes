\documentclass[12pt, letterpaper]{article}
\usepackage[utf8]{inputenc}
\usepackage[left = 2.5cm, right = 2.5cm, top = 3cm, bottom = 3cm]{geometry}
\usepackage{amsthm}
\usepackage{amsfonts}
\usepackage{amsmath}
\usepackage{amssymb}
\usepackage{graphicx}
\usepackage[T1]{fontenc}
\usepackage{proof}

\author{López Soto Ramses Antonio \\
        Quintero Villeda Erik}

\title{Práctica 2: El Lenguaje EAB \\
       {\small Lenguajes de Programación I}}

\date{13 de septtiembre de 2019}

\begin{document}
\maketitle

    \subsection*{Objetivo}
    Implementar la semántica del lenguaje EAB en el lenguaje de programación \textit{Haskell}

    \section*{Introducción}

    \subsection*{Semántica dinámica}
    Las reglas de la semántica dinámica son las siguientes:\vspace{.2cm}

    - Estados\vspace{.5cm}

    \infer[vtrue]{bool[true] {\hspace{.1cm} valor}}{}\vspace{.5cm}
    
    \infer[vfalse]{bool[false] {\hspace{.1cm} valor}}{}\vspace{.5cm} 
    
    \infer[vnum]{num[n] {\hspace{.1cm} valor}}{} \vspace{.5cm}
    
    \infer[edo]{t {\hspace{.1cm} estado}}{t {\hspace{.1cm} asa}} \vspace{.5cm}

    - Estados iniciales\vspace{.5cm}

     \infer[ein]{t {\hspace{.1cm} inicial}}{t {\hspace{.1cm} asa \hspace{.1cm} FV(t) = \emptyset}} \vspace{.5cm}

     - Estados finales\vspace{.5cm}

     \infer[vtrue]{bool[true] {\hspace{.1cm} valor}}{}\hspace{.2cm} \infer[vfalse]{bool[false]{\hspace{.1cm} valor}}{}\hspace{.2cm} \infer[vnum]{num[n] {\hspace{.1cm} valor}}{} \vspace{.5cm}

     \infer[ftrue]{bool[true] {\hspace{.1cm} final}}{}\hspace{.2cm} \infer[ffalse]{bool[false] {\hspace{.1cm} final}}{} \hspace{.2cm} \infer[fnum]{num[n] {\hspace{.1cm} final}}{} \vspace{.5cm}

     - Jucios de las expresiones aritméticas\vspace{.5cm}

     \infer[esumaf]{suma(num[n],num[m]) \rightarrow num[n + m]}{} \vspace{.5cm}

     \infer[eprodf]{prod(num[n],num[m]) \rightarrow num[n * m]}{} \vspace{.5cm}

     \infer[esumai]{suma(t_1,t_2]) \rightarrow suma(t_1',t_2)}{t_1 \rightarrow t_1'} \vspace{.5cm}

     \infer[esumad]{suma(num[n],t_2]) \rightarrow suma(num[n],t_2')}{t_2 \rightarrow t_2'} \vspace{.5cm}

     \infer[eprodi]{prod(t_1,t_2]) \rightarrow prod(t_1',t_2)}{t_1 \rightarrow t_1'} \vspace{.5cm}

     \infer[esumad]{prod(num[n],t_2]) \rightarrow prod(num[n],t_2')}{t_2 \rightarrow t_2'} \vspace{.5cm}

     \infer[esucn]{suc(num[n])\rightarrow num[n+1]}{}\vspace{.5cm}

     \infer[epred0]{pred(num[0])\rightarrow num[0]}{}\vspace{.5cm}

     \infer[epreds]{pred(num[n+1])\rightarrow num[n]}{}\vspace{.5cm}

     \infer[esuc]{suc(t)\rightarrow suc(t')}{t\rightarrow t'}\vspace{.5cm}

     \infer[epred]{pred(t)\rightarrow pred(t')}{t\rightarrow t'}\vspace{.5cm}

     - Jucios de expresiones booleanas\vspace{.3cm}

     \infer[eiftrue]{if(boo[true],t_2,t_3)\rightarrow t_2}{} \vspace{.5cm}

     \infer[eiffalse]{if(boo[false],t_2,t_3)\rightarrow t_3}{} \vspace{.5cm}

     \infer[eif]{if(t_1,t_2,t_3)\rightarrow if(t_1',t_2,t_3)}{t_1\rightarrow t_1'} \vspace{.5cm}

     \infer[]{and(t_1,t_2) \rightarrow and(t_1',t_2)}{t_1\rightarrow t_1'}\vspace{.5cm}

     \infer[]{and(bool[true],t_2) \rightarrow and(bool[true],t_2')}{t_2\rightarrow t_2'}\vspace{.5cm}

     \infer[eiand]{and(bool[false],t_2) \rightarrow bool[false])}{}\vspace{.5cm}

     \infer[]{or(t_1,t_2) \rightarrow or(t_1',t_2)}{t_1\rightarrow t_1'}\vspace{.5cm}

     \infer[]{or(bool[true],t_2) \rightarrow bool[true]}{}\vspace{.5cm}

     \infer[]{not(t) \rightarrow not(t')}{t_1\rightarrow t_1'}\vspace{.5cm}

     \infer[]{not(bool[false]) \rightarrow bool[true])}{}\vspace{.5cm}

     \infer[]{not(bool[true]) \rightarrow bool[false]}{}\vspace{.5cm}

     \infer[]{lt(t_1,t_2) \rightarrow lt(t_1',t_2)}{t_1\rightarrow t_1'}\vspace{.5cm}

     \infer[]{lt(num[n],t_2) \rightarrow lt(num[n],t_2')}{t_2\rightarrow t_2'}\vspace{.5cm}

     \infer[]{l(num[n],num[m]) \rightarrow bool[n < m])}{}\vspace{.5cm}

     \infer[]{gt(t_1,t_2) \rightarrow gt(t_1',t_2)}{t_1\rightarrow t_1'}\vspace{.5cm}

     \infer[]{gt(num[n],t_2) \rightarrow gt(num[n],t_2')}{t_2\rightarrow t_2'}\vspace{.5cm}

     \infer[]{gt(num[n],num[m]) \rightarrow bool[n > m])}{}\vspace{.5cm}

     \infer[]{eq(t_1,t_2) \rightarrow eq(t_1',t_2)}{t_1\rightarrow t_1'}\vspace{.5cm}

     \infer[]{eq(num[n],t_2) \rightarrow eq(num[n],t_2')}{t_2\rightarrow t_2'}\vspace{.5cm}

     \infer[]{eq(num[n],num[m]) \rightarrow bool[n = m])}{}\vspace{.5cm}

     - Jucios de let\vspace{.3cm}

     \infer[eletf]{let(v,x.e_2) \rightarrow e_{2[x:=v]}}{v {\hspace{.5cm} valor}}\vspace{.5cm}

     \infer[eleti]{let(t_1,x.t_2)\rightarrow let(t_1',x.t_2)}{t1\rightarrow t_1'}\vspace{.5cm}

     \subsection*{Semántica estática}
     Las reglas de la semántica estática son las siguientes 

     - Tipado de variables\vspace{.5cm}

     \infer[tvar]{\Gamma,x: T \vdash x: T}{} \vspace{.5cm}

     - Tipado de valores numéricos\vspace{.5cm}

     \infer[tnum]{\Gamma \vdash num[n]:Nat}{}\vspace{.5cm}

     -Tipado de valores booleanos\vspace{.5cm}

    
     \infer[ttrue]{\Gamma \vdash bool[true]:Bool}{} \vspace{.5cm}

     \infer[tfalse]{\Gamma \vdash bool[false]:Bool}{} \vspace{.5cm}

     - Tipado de expresiones aritméticas\vspace{.5cm}

    
     \infer[tsum]{\Gamma \vdash suma(t_1,t_2):Nat}{{\Gamma \vdash t_1 : Nat} \hspace{.1cm} {\Gamma \vdash t_2 : Nat}}\vspace{.5cm}

     \infer[tprod]{\Gamma \vdash prod(t_1,t_2):Nat}{{\Gamma \vdash t_1 : Nat} \hspace{.1cm} {\Gamma \vdash t_2 : Nat}}\vspace{.5cm}
    
     \infer[tsuc]{\Gamma \vdash suc(t): Nat}{\Gamma \vdash t:Nat}\vspace{.5cm}

     \infer[tpred]{\Gamma \vdash pred(t): Nat}{\Gamma \vdash t:Nat}\vspace{.5cm}

     - Tipado de espresiones booleanas\vspace{.5cm}

     \infer[tand]{\Gamma \vdash and(t_1,t_2):Bool}{{\Gamma \vdash t_1 : Bool} \hspace{.1cm} {\Gamma \vdash t_2 : Bool}}\vspace{.5cm}

     \infer[tor]{\Gamma \vdash or(t_1,t_2):Nat}{{\Gamma \vdash t_1 : Bool} \hspace{.1cm} {\Gamma \vdash t_2 : Bool}}\vspace{.5cm}

     \infer[tnot]{\Gamma \vdash not(t): Bool}{\Gamma \vdash t:Bool}\vspace{.5cm}

     \infer[tlt]{\Gamma \vdash lt(t_1,t_2):Bool}{{\Gamma \vdash t_1 : Nat} \hspace{.1cm} {\Gamma \vdash t_2 : Nat}}\vspace{.5cm}

     \infer[tgt]{\Gamma \vdash gt(t_1,t_2):Bool}{{\Gamma \vdash t_1 : Nat} \hspace{.1cm} {\Gamma \vdash t_2 : Nat}}\vspace{.5cm}

     \infer[teq]{\Gamma \vdash eq(t_1,t_2):Nat}{{\Gamma \vdash t_1 : Nat} \hspace{.1cm} {\Gamma \vdash t_2 : Nat}}\vspace{.5cm}

     \infer[tif]{\Gamma \vdash if(t_1,t_2,t_3):T}{{\Gamma \vdash t_1 : Bool} \hspace{.1cm} {\Gamma \vdash t_2 : T} \hspace{.1cm} {\Gamma \vdash t_3 : T}}\vspace{.5cm}

     - Tipado de expresiones let\vspace{.5cm}
    
     \infer[tlet]{\Gamma \vdash let(e_1,x.e_2):S}{{\Gamma \vdash e_1 : T} \hspace{.1cm} {\Gamma, x:T \vdash e_2 : T}}

     \section*{Desarollo}
     La implenentación de las relgas se realizó a través del lenguaje de programación Haskell.\vspace{.3cm}

     En el código se visualizará de una mejor manera cómo se hizo.

     \section*{Conclusión}
     La implenentación de la semántica de la EAB fue muy problemática pues cada que lograbamos que una función se ejecutara 
     correctamente, salían más y más casos en los cuales se ciclaba o no lanzara el resultado esperado. La función con la 
     cual ésto sucedió más veces fue con evals pues sólo nos devolvía los casos en los cuales la entrada era correcta, pero en
     otros casos se ciclaba.\vspace{.3cm}

     En resumen, la práctica fue mucho más compleja en resolver los errores de ejecucion.\vspace{2cm}




     \textbf{El archivo Semantic.hs sí funciona pero al hacer la llamada con make desde el archivo raíz marca un error que no supimos arreglar.
     Se debe ejecutar en el directorio BAE/src.}









 
\end{document}