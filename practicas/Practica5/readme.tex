\documentclass[12pt, letterpaper]{article}
\usepackage[utf8]{inputenc}
\usepackage[left = 2.5cm, right = 2.5cm, top = 3cm, bottom = 3cm]{geometry}
\usepackage[T1]{fontenc}
\usepackage{amsthm}
\usepackage{amsfonts}
\usepackage{amsmath}
\usepackage{amssymb}

\title{Práctica 5: Implementción de MiniC \\
       {\small Lenguajes de Programación I}}

\author{López Soto Ramses Antonio \\
        Quintero Villeda Erik}

\begin{document}
    \maketitle

    \section*{Introducción}

     \subsection*{Objetivo}
     Extender \textit{BAE} para que se comporte como un \textit{lenguaje de programación imperativo}.

    \section*{Desarrollo}
    Los módulos implementados anteriormente que fueron extendidos con las funciones adaptadas:

        \begin{itemize}
            \item Sintax
                  
                \begin{itemize}
                    \item frVars :: Expr -> [Identifier]
                    \item subst :: Expr -> Substitution -> Expr
                \end{itemize}

            \item Dynamic
            
                \begin{itemize}
                    \item eval1 :: (Memory, Expr) -> (Memory, Expr)
                    \item evals :: (Memory, Expr) -> (Memory, Expr)
                    \item evale :: Expr -> Expr
                \end{itemize}

        \end{itemize}

    También se implementó un nuevo módulo \textit{Memory} con las siguientes funciones:

        \begin{itemize}
            \item newAddress :: Memory -> Expr
            \item access :: Address -> Memory -> Maybe Value
            \item update :: Cell -> Memory -> Maybe Value
        \end{itemize}

    Para mayor detalle, véase la implemetación en el directorio BAE/src/BAE/.

    \section*{Conclusión}
    La práctica nos tomó más tiempo de lo esperado pues, como en prácticas pasadas,
    el módulo \textit{Dynamic} nos causó muchos problemas. Al extender \textit{BAE} haciendo uso
    del módulo  \textit{Memory}, no se reconocía éste, mandaba errores que no entendíamos, 
    la extensión nos costó más trabajo de implementar, pero todas las funciones se pudieron hacer
    casi en su totalidad. A diferencia de \textit{Dynamic}, los módulos \textit{Sintax y Memory} fueron muy sencillos,
    pues sólo bastaba con gregar algunas cosas y/o implementar cosas muy rápido. \vspace{.3cm}

    En conclusión, El objetivo inicial de la práctica fue lcanzado, pues se implementó una simulación 
    (pequeña) de lo que es un \textit{Leguaje de programación impertivo}  
\end{document}

