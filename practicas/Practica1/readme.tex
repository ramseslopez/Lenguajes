\documentclass[12pt, letterpaper]{article}
\usepackage[utf8]{inputenc}
\usepackage[spanish]{babel}
\usepackage[left = 2.5cm, right = 2.5cm, top = 3cm, bottom = 3cm]{geometry}
\usepackage{amsthm}
\usepackage{amsfonts}
\usepackage{amsmath}
\usepackage{amssymb}
\usepackage{graphicx}
\usepackage[T1]{fontenc}

\author{López Soto Ramses Antonio \\
        Quintero Villeda Erik}

\title{Práctica 1: Primer intérprete de Expresiones Aritmético-Booleanas \\
       {\small Lenguajes de Programación I}}

\date{21 de agosto de 2019}

\begin{document}
\maketitle

  \section*{Introducción}

  \subsection*{Objetivo}
  Implementar un intérprete de expresiones artimético-booleanas (\textbf{EAB})
  a través del lenguaje de programación \textit{Haskell}.\vspace{.3cm}

  PostFix se trata de un lenguaje simple basado en una pila e inspirado en el
  lenguaje gráfico PostScript, el lenguaje de programación Forth y las
  calculadoras HP.

  \section*{Desarrollo}
  La implementación del intérprete se realizo por medio del lenguaje de
  programación \textit{Haskell} a manera de simular el mismo comportamiento del
  lenguaje \textit{Postfix}

  \subsection*{Implementación}
  Para la implementación del intérprete se utilizó lo siguientes:\vspace{.2cm}


  \hspace{.5cm} data Instruction = I Int | B Bool \vspace{.1cm}

  \hspace{4cm}                  | ADD | AND | OR | DIV | EQQ | EXEC \vspace{.1cm}

  \hspace{4cm}                  | GET | Gt | Lt | POW | MAX | MIN | FACT | MUL \vspace{.1cm}

  \hspace{4cm}                  | NOT | POP | REM | SEL | SUB \vspace{.1cm}

  \hspace{4cm}                  | SWAP | ES [Instruction] deriving (Eq, Show) \vspace{.2cm}


  \hspace{.5cm} type Stack = [Instruction]\vspace{.2cm}

  \hspace{.5cm} type Program = [Instruction]\vspace{.2cm}

  \hspace{.5cm} data Expr = N Int | T | F | Succ Expr | Pred Expr \vspace{.1cm}

  \hspace{4cm}             | Expr :+ Expr | Expr :− Expr | Expr :∗ Expr \vspace{.1cm}

  \hspace{4cm}             | Expr :/ Expr | Expr :\% Expr | Not Expr  \vspace{.1cm}

  \hspace{4cm}             | Expr :& Expr | Expr :| Expr \vspace{.1cm}

  \hspace{4cm}             | Expr :> Expr | Expr :< Expr | Expr := Expr \vspace{.1cm}

  \hspace{4cm}             | Expr :ˆ Expr| Max Expr Expr | Min Expr Expr \vspace{.1cm}

  \hspace{4cm}             | Fact Expr deriving (Eq,Show) \vspace{.3cm}

  Las funciones implementadas en el archivo \textit{practica1.hs} fueron Los
  siguientes:\vspace{.3cm}

  \begin{itemize}
    \item \textit{Función que se encarga de realizar todas las operaciones
                  aritméticas haciendo uso de instrucciones.}\vspace{.05cm}

                  arithOperation :: Instruction -> Instruction -> Instruction -> Instruction

    \item \textit{Función que se encarga de mostrar la valuación del operador
                  binario AND a modo de instrucción}\vspace{.05cm}

                  bboolOperation :: Instruction -> Instruction -> Instruction -> Instruction

    \item \textit{Función que se encarga de mostrar la valuación del operador
                  unario NOT a modo de instrucción}\vspace{.05cm}

                  uboolOperation :: Instruction -> Instruction -> Instruction

    \item \textit{Función que se encarga de mostrar el resultado luego de
                    comparar si dos números son iguales o uno es mayor o menor
                    que otro.}\vspace{.05cm}

                    relOperation :: Instruction -> Instruction -> Instruction -> Instruction

    \item \textit{Función que se encarga de realizar las operaciones de una
                    pila usando las instrucciones definidas}\vspace{.05cm}

                    stackOperation :: Stack -> Instruction -> Stack

    \item \textit{Función que devuelve la lista de instrucciones y la pila}\vspace{.05cm}

                    execOperation :: [Instruction] -> Stack -> Instruction -> ([Instruction], Stack)

    \item \textit{Función que se encarga de de realizar la evaulación de un
                    programa y una lista de instrucciones}\vspace{.05cm}

                    executeProgram :: Program -> Stack -> Stack

    \item \textit{Función que se encarga de traducir una istrucción a un programa}\vspace{.05cm}

                    compile:: Expr -> Program

    \item \textit{Función que se encarga de traducir una expresión a una instrucción}\vspace{.05cm}

                    execute :: Expr -> Instruction

  \end{itemize}

  A lo largo de la práctica se hizo uso de la recursión para definir todas
  y cada una de las funciones requieridas.

  \section*{Conclusión}
  La implementación del intérprete fue un tanto complicada. Acabamos de
  implementar el intérprete con tiempo, pero al verificarla, notamos que
  algunas funciones se encontraban mal implementadas, ya que, nosotros definimos
  las funciones de una forma, la cual, no era correcta. Por lo que tuvimos que
  volver a implementar y arreglar algunas cosas.\vspace{.3cm}

  Simular el comportamiento de PostFix a través de este intérprete nos ayudó a
  comprender de una manera mucho más clara su comportamiento, y además, a tener
  el cuidado de verificar si la implementación es la correcta o se debe de
  corregir y/o mejorar, si es posible.\vspace{.3cm}

  En conclusión, lo que esta práctica tiene como fin, es que el lenguaje PostFix
  sea comprendido de manera más fácil por medio de una implementación, la cual
  no es idéntica al lenguaje pero si cercana para comprender su comportamiento.



\end{document}
